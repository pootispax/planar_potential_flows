\section{Functional requirement of the program}
\subsection{The project}
The goal of this project is to simulate the movement of a fluid through
different geometries. The program create a box of choosen size, builds a
geometry inside it and simulates the movement of a given fluid.

\subsection{The files}
In order to increase readability, the project is made of several files.
I made the choice to work with Object Oriented Programming.
\begin{itemize}
    \item main.py: This file calls for the needed functions/class
    \item matrices.py: This file contains the class ``Matrices'', it builds the
          geometry, the different matrices to plot and stores them
    \item plot.py: This file plots the matrices built in ``matrices.py''
    \item parameters.py: This file contains all the variables that can be 
          changed by the user
    \item data\_check.py: This file checks the variables and makes sure that
          the program will run
\end{itemize}

\subsection{The data}
This project uses several piece of data set by the user to work.
\begin{itemize}
    \item $N_x \text{ and } N_y$ are the size of the domain
    \item $h$ represents the size of a cell
    \item $geometry$ corresponds to the choosen geometry
    \item $angle$ corresponds to the angle of the widening/shrinkage geometry
    \item $v_x$ is the Neuman condition
    \item $\phi_{ref}$ is the Dirichlet condition
\end{itemize}
Be careful in the case of a widening/shrinkage geometry !
In order for the program to generate a domain from one end to another, there is
a restriction on the angle, if the restriction is not met, the program will
output a ValueError. The restriction is as follows:
\[
      |angle| < \arctan{\left(\dfrac{0.5 \times N_y - 1}{N_x}\right)}
\]
The angle parameter should be set in degree, the program will convert it to
radians for the computation.

\subsection{The outputs}
As of the alpha version, the program outputs 4 pdf files, one for each plot.
The filenames are set with the following rule:\\
\begin{center}
      \mintinline{python}{data\_geometry\_Nx\_Ny.pdf}
\end{center}

